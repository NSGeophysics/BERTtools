\documentclass[11pt]{article}

\usepackage{url}
\setlength{\parskip}{0.5cm plus4mm minus3mm}

\textwidth=6.4in
\textheight=8.5in
\hoffset=-0.7in
\voffset=-0.7in

\setlength{\parindent}{0cm} 

%\usepackage{fullpage}


\title{BERT tutorial}

\begin{document}

\maketitle

\section{Introduction}
BERT is a powerful 2-D and 3-D electrical resistivity inversion
program and it is relatively easy to use. It is installed on the
Student Linux computer.  You can use it either by physically sitting
at the Linux computer or by accessing your account remotely.


\section{Workflow}
Using BERT for basic inversions is generally very simple. It happens
in 3 steps:
\begin{enumerate}
\item Create an inversion configuration file (and edit it if you need to)
\item run the calculations
\item create a .vtk file that you can look at using paraview
\end{enumerate}

\section{Preparation and editing of inversion file}
To do step 1, type, in the Linux command line (which you openend
through PuTTY)

\quad \verb#bertNew3D datafile.ohm > invfile.cfg#

Here you should replace \verb#datafile.ohm# with the name of the .ohm
datafile you are using and replace \verb#invfile.cfg# with the name
you would like to use for your inversion file. If you would like to
run 2.5-D inversion instead of inversion, replace \verb#bertNew3D#
with \verb#bertNew2D#.

Once the inversion file is created, you can copy it to your machine
and have a look at the parameters or use emacs to look at it directly
through the command line:

\quad \verb#emacs invfile.cfg#

Which will open the text file. To save changes, hold Ctrl and press
``x'' and then ``s''. To exit the editing screen and return to command
line hold Ctrl and then press ``x'' and then ``c''.

\section{Running the inversion}

To do the calculations, run

\quad \verb#bert invfile.cfg all#

This may take a while as such calculations can be quite involved. In
particular if you edited the inversion configuration file to make a
fine mesh.

\section{Creating visualization files}
To create visualization output files (.vtk which you can view using
paraview), run

\quad \verb#bert invfile.cfg show#

This will create a .vtk file which you can now copy onto your computer
using PSCP.

To view the .vtk file on your computer, use Paraview which you can
obtain from here: \url{www.paraview.org/}

\section{Storing your calculation results}
To store your calculation results, for example if you want to run
another calculation using different settings but don't want to
overwrite your results, run

\quad \verb#bert invfile.cfg save myfoldername#

This will create a new folder with the name \verb#myfoldername# and
automatically copy all relevant files into it. You can change the name
of the folder if you like.

\section{Cleaning up calculation results}

To clean up the folder from temporary calculation results, run

\quad \verb#bert invfile.cfg clean#

This will keep your data and cfg files and only delete temporary
calculation files.

\section{More options}

BERT has many more useful commands. Run

\qquad \verb+bert+

to get a list of commands. Also, read through the BERT tutorial\\
\url{http://www.resistivity.net/download/bert-tutorial.pdf}.

\end{document}

